\documentclass[xcolor=dvipsnames]{beamer}
\usetheme[secheader]{Boadilla}
%\usetheme[secheader]{Madrid}
\usecolortheme[named=ForestGreen]{structure} 
\setbeamertemplate{items}[ball] 
\setbeamertemplate{blocks}[rounded][shadow=true] 

\usepackage[utf8]{inputenc}
\usepackage{algorithmic}
\usepackage[procnames]{listings}
\usepackage{color}
\usepackage{verbatim}
\usepackage{graphicx}
\usepackage{subfigure}
\usepackage{qtree}


\title[Database Management Systems]{
	Database Management Systems \\
	{\normalsize Cap 1. Introducere. Modelul Relaţional}
}

\author[]{}
\date{\today}
\institute[]{}

\begin{document}

\definecolor{keywords}{RGB}{255,0,90}
\definecolor{comments}{RGB}{0,0,113}
\definecolor{red}{RGB}{160,0,0}
\definecolor{green}{RGB}{0,100,0}

\definecolor{dkgreen}{rgb}{0,0.6,0}
\definecolor{gray}{rgb}{0.5,0.5,0.5}
\definecolor{mauve}{rgb}{0.58,0,0.82}

\lstset{language=Python, 
	    basicstyle=\ttfamily\small, 
	    keywordstyle=\color{keywords},
	    commentstyle=\color{comments},
	    stringstyle=\color{red},
	    showstringspaces=false,
	    identifierstyle=\color{green},
	    procnamekeys={def,class}}
	    
\lstset{language=Java,
	  aboveskip=3mm,
	  belowskip=3mm,
	  showstringspaces=false,
	  columns=flexible,
	  basicstyle={\small\ttfamily},
	  numbers=none,
	  numberstyle=\tiny\color{gray},
	  keywordstyle=\color{blue},
	  commentstyle=\color{dkgreen},
	  stringstyle=\color{mauve},
	  breaklines=true,
	  breakatwhitespace=true,
	  tabsize=3
}
\lstset{language=SQL,
      showspaces=false,
      basicstyle=\ttfamily,
      numbers=left,
      numberstyle=\tiny,
      keywordstyle=\color{blue},
      commentstyle=\color{gray},
      commentstyle=\color{dkgreen},
      stringstyle=\color{mauve}
}



\begin{frame}
\titlepage
\end{frame}

\frame{
	\tableofcontents
}

\section{Introducere}
\frame{\tableofcontents[currentsection]}

\frame {
	\frametitle{Database Management System (DBMS)}
	\begin{itemize}
	  \item manipularea unor cantităţi mari de date, mai exact \ldots
	  \item o modalitate eficientă, sigură, uşor de programat, cu
	  acces multi-user, de stocare şi accesare a unor volume masive de date
	  persistente
	  \item larg răspândite: web-sites, sisteme bancare, sisteme de comunicaţie,
	  experimente ştiinţifice
	\end{itemize}
}


\frame{
	\frametitle{Queries}
	\begin{minipage}[t]{1\linewidth}
		\begin{itemize}
		  \item un query (o expresie) aplicat(ă) pe o mulţime de relaţii va avea ca
		  rezultat o altă relaţie
		\end{itemize}
	\end{minipage}
	\vfill
	\begin{minipage}[t]{1\linewidth}
	
	\end{minipage}	
}


\frame{
	\frametitle{Operatorul 'project'}
	\only<1> {
		\begin{itemize}
		  	\item alege anumite coloane
		  	\item ID-urile studenţilor şi deciziile lor
		  	\item[] $\pi_{sID, decision} Apply$ - doar două coloane
		\end{itemize}
	}
	\only<2>{
		\begin{itemize}
		  	\item ID-urile studenţilor şi numele lor, pentru cei cu $GPA > 3.7$
		  	\item compunerea operatorilor:
		\end{itemize}		
	}
	\only<3>{
		\begin{itemize}
		  	\item ID-urile studenţilor şi numele lor, pentru cei cu $GPA > 3.7$
		  	\item compunerea operatorilor:
	  		\item[] $\sigma_{GPA > 3.7}$
		\end{itemize}		
	}
	\only<4>{
		\begin{itemize}
		  	\item ID-urile studenţilor şi numele lor, pentru cei cu $GPA > 3.7$
		  	\item compunerea operatorilor:
	  		\item[] $\pi_{sID, name}(\sigma_{GPA > 3.7})$
	  		\item select şi project lucrează nu neapărat pe o relaţie, ci pe orice
	  		expresie
		\end{itemize}		
	}	
	\begin{minipage}[t]{1\linewidth}
		
	\end{minipage}
}

\frame{
	\frametitle{Exemplu: select şi project}
	\begin{itemize}
  		\item e util să compunem două select-uri ? exemplu:
  		\item[] $\sigma_{state='CA'}(\sigma_{enroll > 1000} College)$ \only<2->{NU}
  		\item<3-> dar două project-uri ? exemplu:
  		\item<3->[] $\pi_{GPA}(\pi_{sID, GPA, sizeHS} Student)$ \only<4->{NU}
  		\item<5-> dar e permis? \only<6->{DA} 
	\end{itemize}	
	\begin{minipage}[t]{1\linewidth}
		
	\end{minipage}
}



\end{document}


